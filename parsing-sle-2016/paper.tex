\documentclass[nocopyrightspace,10pt]{sigplanconf}
% \documentclass[nocopyrightspace,10pt]{sig-alternate-05-2015}
\newif\ifdraft
\draftfalse
\newif\ifsigplanconf
\sigplanconftrue
\newif\ifsubmission
\submissionfalse

\usepackage{times-lite}
\usepackage{mathptm}
\usepackage{txtt}
\usepackage{stmaryrd}
%\usepackage{ttquot}
%\usepackage{angle}
\usepackage{proof}
\usepackage{bcprules}
\usepackage{amsmath}
\usepackage{amssymb}
%\usepackage{amsthm}
\usepackage{afterpage}
\usepackage{balance}
\usepackage{color}
\usepackage{listings}
\usepackage{xspace}

\newcommand\secref[1]{Section~\ref{#1}}
\newcommand\figref[1]{Figure~\ref{#1}}

%\newcommand\todo[1]{{\color{red}{[\textsf{TODO:} \emph{#1}]}}}
\newcommand\todo[1]{}
\newcommand\eat[1]{}

\renewcommand\c[1]{\normalfont{\texttt{#1}}}
\renewcommand\k[1]{\texttt{\textbf{#1}}}
\newcommand\e[1]{\emph{#1}}

\newcommand{\bnf}{\ensuremath{\ \ | \ \ }}
\newcommand{\ty}{\ensuremath{\!:\!}}

\newcommand\stepsone{\rightarrow}
\newcommand\stepsmany{\rightarrow^*}
\newcommand\effect[1]{\xrightarrow{#1}}
\newcommand\xstepsone[1]{\xrightarrow{#1}}
\newcommand\xstepsmany[1]{\xrightarrow{#1}{\!\!}^*\;}

\newcommand\Lb{\llbracket}
\newcommand\Rb{\rrbracket}
\newcommand\SB[1]{\Lb#1\Rb} % put double brackets around the argument
\newenvironment{SBE}{\left\Lb\begin{array}{c}}%
{\end{array}\right\Rb}

\newcommand\many[1]{\overline{#1}}
\newcommand\lift[1]{\lfloor{#1}\rfloor}


\usepackage{graphicx}

\ifdraft
\usepackage{layout}
\fi

\usepackage{ifpdf}

% make it A4 despite ACM 
% \ifpdf
% \setlength{\pdfpagewidth}{210mm}
% \setlength{\pdfpageheight}{297mm}
% \fi

\hyphenation{name-space}
\hyphenation{name-spaces}

% redefine \andalso from bcprules to tighten up space
\renewcommand{\andalso}{\quad}

%\lefthyphenmin=1
%\righthyphenmin=1


\begin{document}

\eat{
\conferenceinfo{PPoPP 2014} {February 2014, Orlando, Florida.}
\copyrightyear{2013}
\copyrightdata{}
}

\title{There's more than one way to skin a cat}

\ifsigplanconf

\authorinfo{Nathaniel Nystrom}
   {Faculty of Informatics \\ Universit\`a della Svizzera italiana \\
   Lugano, Switzerland}
   {nate.nystrom@usi.ch}

   \else

\numberofauthors{1} 
\author{
\alignauthor
  Nathaniel Nystrom \\
  \affaddr{Faculty of Informatics} \\
  \affaddr{Universit\`a della Svizzera italiana} \\
  \affaddr{Lugano, Switzerland} \\
  \email{nate.nystrom@usi.ch}
}

\fi

\vspace{-1cm}
\maketitle

\sloppy

\bibliographystyle{plain}

\begin{abstract}
  When developing a new programming language,
  writing a parser
  is a necessary, but often tedious, task.
  During development, a language developer may want to experiment with
  different styles of concrete syntax: should the language be
  indentation-based
  like Python or Haskell,
  or should it belong to the curly-brackets family of languages
  (C, Java, etc.)?

  Skinner is a tool that automatically generates a parser directly 
  from abstract-syntax-tree definitions, using a \emph{language skin}
  to seed the parser generator with the appropriate syntax.
  The language skin includes a grammar template, a
  description of the lexical and syntactic features of the skin language.
  For instance, a Python language skin
  would contain a template for a scanner and parser for Python.
  Skinner tries to match the AST types to
  constructs in the grammar, using existing rules,
  removing unused rules, and
  creating new rules as necessary to instantiate a parser that generates
  the given ASTs. The user can use
  the generated parser as is, or modify it to taste.

\end{abstract}

\section{Language skins}
\label{sec:intro}

A language skin contains a template grammar and other declarations
needed to describe the language used to seed the parser generator.
The template grammar
is simply a context-free grammar in which
each
rule has an associated semantic action that
invokes a factory method to create the abstract syntax.
The skin provides an interface
but no implementation of these factory methods.
Rules and factory methods are typed using AST interface types declared
in the skin.

To subjectively improve the quality
of the generated parser,
the grammar
template can be augmented with additional rules in the style of the 
skin language.
For instance, although Java does not have tuples,
the Java skin contains additional rules to parse tuples.
% Using these rules,
% the generated parser can handle tuple-like constructs.

\section{The skinning process}
\label{sec:matching}

Matching AST classes to the template grammar is done as follows.
The AST types and the types in the skin are matched
by name.
Rather than performing an exact match, names are matched approximately.
The skin specifies sets of aliases
to allow for divergent naming schemes. For example, since ``plus''
and ``add'' are considered aliases, a
\c{PlusExp} AST class can be matched with the \c{Add} skin interface type.

Expressions are generated to create instances of the AST classes.
For instance, given the following AST classes:
\begin{verbatim}
abstract class Exp
class Value(int value) extends Exp
class Bin(Op op, Exp left, Exp right) extends Exp
enum Op { Plus, Minus }
\end{verbatim}
the following instantiation expressions are generated:
\begin{verbatim}
new Value(value)
new Bin(Plus, left, right)
new Bin(Minus, left, right)
\end{verbatim}
These expressions are then matched against the factory method interface
in the skin. The parameters of a matched factory method are renamed
to provide bindings for each of
free variables in the instantiation expression. Coercions
are generated if necessary to pass values of the correct type. 
Since there may be many factory methods with the same type,
the matching process also uses the names of the methods and of the AST
classes to find an appropriate match, similar to how types are matched. 
For example, the \c{IntLit} factory method might be matched against the
\c{Value} expression above, resulting in the following implementation:
\begin{verbatim}
Exp IntLit(int v) { return new Value(v); }
\end{verbatim}

New grammar rules are generated for any unmatched instantiation expression.
The new rule is based
on rules in the template grammar.
A left-hand-side nonterminal of the correct type is selected and the rule's
right-hand-side is generated to provide semantic values of the appropriate
types, using existing grammar symbols. The AST class name is used to create a new keyword, which is added as
a token to the scanner.

The resulting grammar is pruned to remove unused rules and symbols.
The grammar is then used to generate a GLR parser that creates the AST.

\section{Conclusions}
\label{sec:conclusion}

Language skinning allows a developer to concentrate on the semantics of the
language rather than the concrete syntax. 
We find Skinner to be useful for generating parsers for DSLs and small teaching languages,
and for experimenting with different concrete syntaxes in larger
programming languages.

%\balance
% \bibliography{bib}

% \clearpage

\end{document}
